\documentclass[11pt]{article}
\usepackage[a4paper, margin=2.5cm]{geometry}
\usepackage[slovene]{babel}
\usepackage[utf8]{inputenc}
\usepackage[T1]{fontenc}

% Definicija okolij izrek, posledica
%\theoremstyle{plain}
%\newtheorem{izrek}{Izrek}[section]
%\newtheorem{posledica}[izrek]{Posledica}


% Definicija okoli za definicije in vaje
%\theoremstyle{definition}
%\newtheorem{definicija}[izrek]{Definicija}
%\newtheorem{vaja}[izrek]{Vaja}


\begin{document}

\title{Brownovo gibanje}
\author{Matej Rojec}
\date{}
\maketitle

Brownovo gibanje (več v !!) je intuitivno slučajen proces, % Sklic na knjigo
ki predstavlja naključno gibanje delcev v mediju.
    
    % Slika: PerrinPlot2.pdf
    % Napis pod sliko: 
    % Reprodukcija slike iz Jean Baptiste Perrin, \emph{Mouvement brownien et réalité moléculaire}, Ann. de Chimie et de Physique (VIII) 18, 5-114, 1909

    % Začetek definicije
    Standardno Brownovo gibanje $\{B_t\}_{t \geq 0}$ je slučajen proces z naslednjimi lastnostmi: 
        $B_0 = 0$.
        Prirastki $B_{t_n} - B_{t_{n-1}}, B_{t_{n-1}} - B_{t_{n-2}}, \ldots, B_2 - B_1, B_1 - B_0$ so neodvisne slučajne spremenljivke, za vsak $t_0 \leq t_1 \leq \cdots \leq t_{n-1} \leq t_n$.
        Za vsak $t \geq 0$ in $h > 0$ velja $B_{t+h} - B_t \sim \mathcal{N}(0, h)$.
        Funkcija $t \mapsto B_t$ je zvezna skoraj gotovo.
    % Konec definicije
    
    Preden zapišemo izrek, definirajmo še pojem časa ustavljanja.
    
    % Začetek definicije
    Slučajna spremenljivka $\tau$ na verjetnostnem prostoru ?? z vrednostmi v ??
    je čas ustavljanja glede na filtracijo ??, če velja ??.
    % Konec definicije
    
    Zdaj lahko zapišemo izrek !!. % Sklic na izrek z oznako thm:stopped_brownian
    
    % Začetek izreka
    Naj bo $\{B_t\}_{t \geq 0}$?? (standardno) Brownovo gibanje, ?? čas ustavljanja glede na 
    ?? in naj bo ??.
    Potem je tudi proces:
    \[
    \hat{B} := \{B_{T+t} - B_T \mid t \geq 0\}
    \]
    (standardno) Brownovo gibanje in neodvisen od ??.
    % Konec izreka
    
\end{document}